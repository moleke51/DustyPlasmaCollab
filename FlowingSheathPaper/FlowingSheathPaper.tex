\documentclass[journal]{Imperial_lab_report}

\usepackage{here}
\usepackage{float}
\usepackage{adjustbox}
\usepackage{graphicx}
\usepackage{cite}
\usepackage{amsmath}
\usepackage{amssymb}
\usepackage{pifont}
\usepackage{enumitem}
\usepackage{url}
\usepackage{multirow}
\usepackage{etoolbox}
\usepackage{titlesec}
\newcommand{\cmark}{\ding{51}}

\DeclareMathOperator\erf{erf}

\interdisplaylinepenalty=2500
\hyphenation{op-tical net-works semi-conduc-tor}
\patchcmd{\thebibliography}{\section*{\refname}}{}{}{}
\setcounter{secnumdepth}{4}
\titleformat{\paragraph}
{\normalfont\normalsize\bfseries}{\theparagraph}{1em}{}
\titlespacing*{\paragraph}
{0pt}{3.25ex plus 1ex minus .2ex}{1.5ex plus .2ex}


\begin{document}

\title{Potential of a large dust grain in a collisionless plasma}
\author{Dogan Akpinar and George E. B. Doran}
\markboth{D. Akpinar \& G. E. B. Doran }
{Shell \MakeLowercase{\textit{et al.}}:}

\maketitle

\begin{abstract}

\end{abstract}

\section{Introduction}

\section{Radial motion theory (ABR)}
\smallskip

The ABR model is a radial motion theory derived by Allen, Boyd and Reynolds. It describes the equilibrium surface potential reached 
by a dust grain immersed in an infinite and stationary plasma.

\smallskip

Consider a spherical dust grain, of arbitrary size, immersed in this infinite plasma. Far from the surface we assume that the electron
and ion densities are equal; known as quasi-neutrality. 

\section{Modified orbital motion limited (MOML)}
\section{SCEPTIC numerical fit}
\section{Comparison of MOML and ABR with SCEPTIC data}
\section{Flowing sheath approximation}
\section{Conclusion}
\section{References and Acknowledgements}








\bibliography{DustyLib}
\bibliographystyle{IEEEtran}

\section{Appendix}







\end{document}