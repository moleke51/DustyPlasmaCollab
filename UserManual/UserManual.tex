\documentclass{journal}

\usepackage{here}
\usepackage{float}
\usepackage{adjustbox}
\usepackage{makecell}
\usepackage{graphicx}
\usepackage{cite}
\usepackage{amsmath}
\usepackage{amssymb}
\usepackage{pifont}
\usepackage{enumitem}
\usepackage{url}
\usepackage{multirow}
\usepackage{etoolbox}
\usepackage{titlesec}
\usepackage{caption}
\usepackage{changepage}
\usepackage{dsfont}
\usepackage[latin1]{inputenc}
\usepackage{tikz}
\usepackage{array}
\usetikzlibrary{shapes,arrows}
\captionsetup[table]{skip=5pt}
\newcommand{\cmark}{\ding{51}}

\DeclareMathOperator\erf{erf}

\interdisplaylinepenalty=2500
\hyphenation{op-tical net-works semi-conduc-tor}
\patchcmd{\thebibliography}{\section*{\refname}}{}{}{}
\setcounter{secnumdepth}{4}
\titleformat{\paragraph}
{\normalfont\normalsize\bfseries}{\theparagraph}{1em}{}
\titlespacing*{\paragraph}
{0pt}{3.25ex plus 1ex minus .2ex}{1.5ex plus .2ex}


\begin{document}

\title{Dust grain potential calculator \\ \it{User manual}}
\author{Dogan Akpinar and George E. B. Doran}
\markboth{D. Akpinar \& G. E. B. Doran}
{Shell \MakeLowercase{\textit{et al.}}:}

\maketitle

\section{Setup}

\begin{itemize}
    \item Install the modules specified in \textit{requirements.txt}
    \item Open \textit{Dust\_grain\_potential\_calculator.py} and edit the base path on line 8
\end{itemize}

\section{Variables}

\begin{table}[H]
    \begin{adjustwidth}{-2.5cm}{}
        \label{tab:ValueTable}
        \begin{tabular}{|c|c|c|c|c|c|c|} 
        \hline
        Variable name & Unit & Requirements & \makecell{Normalised \\ variable name} & \makecell{Normalisation \\ factor} & Default value & Variable\\
        \hline
        Electron temperature ($T_e$) & $K$ & $T_e > 0$ & - & - & - & Yes \\
        \hline
        Ion temperature ($T_i$) & $K$ & $T_i \geq 0$ & $\Theta$ & $T_e$ & - & Yes \\
        \hline
        Relative ion charge ($z$) & - & \makecell{$0 < z \leq z_{max}$ \\ z $\in \mathds{Z}$} & - & - & - & No \\
        \hline
        Ion mass ($m_i$) & $kg$ & $m_i > 0$ & $\mu^2$ & $m_e$ & - & No \\
        \hline
        \makecell{Electron number density \\ at infinity ($n_0$)} & $m^{-3}$ & $n_0>0$ & - & - & - & No\\
        \hline
        Dust grain radius ($a$) & $m$ & $a \geq 0$ & $\alpha$ & $\lambda_D = \sqrt{\frac{\varepsilon_0 k_B T_e}{n_0 e^2}}$ & - & Yes\\
        \hline
        Flow speed (v) & $ms^{-1}$ & v $\geq 0$ & $\upsilon$ & $\upsilon_B = \sqrt{\frac{z k_B T_e}{m_i}}$ & 0 & Yes \\
        \hline
        \end{tabular}
    \end{adjustwidth}
\end{table}

\section{Running the code}

\medskip

% Define block styles
\tikzstyle{decision} = [diamond, draw, fill=yellow!20, 
    text width=5em, text badly centered, node distance=3cm, inner sep=0pt]
\tikzstyle{block} = [rectangle, draw, fill=blue!20, 
    text width=7em, text centered, rounded corners, minimum height=4em]
\tikzstyle{resultBlock} = [rectangle, draw, fill=red!20, 
    text width=7em, text centered, rounded corners, minimum height=4em]
\tikzstyle{requirement} = [ellipse, draw, fill=green!20, 
    text width=4em, text centered, rounded corners, minimum height=4em]
\tikzstyle{line} = [draw, -latex']
\tikzstyle{cloud} = [draw, ellipse,fill=red!20, node distance=3cm,
    minimum height=2em]
    
\begin{tikzpicture}[node distance = 2cm, auto]
    % Place nodes
    \node [block] (init) {Do you want to use dimensionless variables (y/n)};
    \node [decision, below left of = init, node distance = 4cm] (decideY) {y};
    \node [decision, below right of = init, node distance = 4cm] (decideN) {n};
    \node [block, below of = init, node distance = 5.5cm] (variableInput) {Enter variables according to requirements};
    \node [decision, below right of = variableInput, node distance = 4cm] (variables) {variable};
    \node [decision, below left of = variableInput, node distance = 4cm] (number) {[number]};
    \node [requirement, below of = variables, node distance = 3cm] (variableCentre) {Variable already chosen};
    \node [requirement, left of = variableCentre, node distance = 3cm] (variableLeft) {Can be varied};
    \node [requirement, right of = variableCentre, node distance = 3cm] (variableRight) {Can not be varied};
    \node [block, below right of = variableCentre, node distance = 4cm] (Retry) {Enter variables according to requirements};
    \node [block, below of = variableLeft, node distance = 2.8cm] (variableChosen) {Variable chosen};
    \node [block, below of = variableChosen, node distance = 3cm] (nextVariable) {Repeat for the other variables};
    \node [resultBlock, below of = nextVariable, node distance = 3cm] (result) {Result};
    %\node [block, below right of = variableCentre, node distance = 4cm] (NextVariable) {Enter next variable};
    % Draw edges
    \path [line] (init) -- (decideY);
    \path [line] (init) -- (decideN);
    \path [line] (decideY) -- (variableInput);
    \path [line] (decideN) -- (variableInput);
    \path [line] (variableInput) -- (variables);
    \path [line] (variableInput) -- (number);
    \path [line] (variables) -- (variableCentre);
    \path [line] (variables) -- (variableLeft);
    \path [line] (variables) -- (variableRight);
    \path [line] (variableCentre) -- (Retry);
    \path [line] (variableRight) -- (Retry);
    \path [line] (variableLeft) -- (variableChosen);
    \path [line] (variableChosen) -- (nextVariable);
    \path [line] (number) |- (nextVariable);
    \path [line] (nextVariable) -- (result);
    
    
\end{tikzpicture}

\end{document}